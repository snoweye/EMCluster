
{\color{red} \bf Warning:} This document is written to explain the main
functions of \pkg{EMCluster}~\citep{Chen2012EMClusterpackage}, version 0.2-0.
Every effort will be made to ensure future versions are consistent with
these instructions, but features in later versions may not be explained
in this document.


\section[Introduction]{Introduction}
\label{sec:introduction}
\addcontentsline{toc}{section}{\thesection. Introduction}

We use this section to show how to install \pkg{EMCluster},
introduce basic finite mixture Gaussian distribution,
and notation for the corresponding \proglang{R} objects.
In Section~\ref{sec:em_init}, we introduce EM algorithms and
strategies of initialization for model-based clustering, and
the major \proglang{R} functions are also introduced.
In Section~\ref{sec:example}, we provide two
examples for unsupervised and semi-supervised clusterings,
and quick demos are shown.


\subsection[Installation]{Installation}
The \pkg{EMCluster} has simple interface of \proglang{R} to efficient
\proglang{C} code that we optimize the algorithm and utilize \pkg{LAPACK}
for matrix algebra. The package should install on most popular platform with
further configureations. The installation can be done in the shell command
as
\begin{Command}
> R CMD INSTALL EMCluster_0.2-0.tar.gz
\end{Command}
or from any R session as
\begin{Command}
R> install.packages("EMCluster") 
\end{Command}
with user-favored CRAN mirror site.


\subsection[Notation]{Notation}
The \pkg{EMCluster} assumes finite mixture Gaussian distribution with
unstructured dispersion and implements EM algorithm for model-based clustering
in both unsupervised and semi-supervised clusterings. The model is
\begin{equation}
f(\bx | \bvartheta) = \sum_{k = 1}^K \pi_k \phi(\bx | \bmu_k, \bSigma_k).
\label{eqn:density_mvn}
\end{equation}
\begin{itemize}
\item
  $\bx$ is a $p$-dimensional observation.
\item
  $\bvartheta = \{\pi_1,\pi_2,\ldots,\pi_{K-1},\bmu_1,\bmu_2,\ldots,\bmu_K,
    \bSigma_1,\bSigma_2,\ldots,\bSigma_K\}$.
\item
  $\pi_1,\pi_2,\ldots,\pi_K$ are mixing proportion for $K$ components that
  $\sum_{k = 1}^K \pi_k = 1$ and $0 < \pi_k < 1$ for all $k=1,2,\ldots,K$.
\item
  $\phi(\bx|\bmu_k,\bSigma_k)$'s are multivariate Gaussian densities with mean
  $\bmu_k$ and dispersion $\bSigma_k$.
\item
  $\bmu_1,\bmu_2,\ldots,\bmu_K$ are $p$-dimensional mean vectors and
  $\bSigma_1,\bSigma_2,\ldots,\bSigma_K$ are $p\times p$-dimensional
  dispersion matrices.
\end{itemize}

We also assume the following notation for \proglang{R} objects
in \pkg{EMCluster}
\begin{itemize}
\item
  \code{x} is a matrix with dimension $n\times p$
  containing $n$ observations to be clustered.
\item
  \code{pi} is a vector with length $K$ containing $\pi_1,\pi_2,\ldots,\pi_K$.
\item
  \code{Mu} is a matrix with dimension $K\times p$ containing
  $\bmu_1,\bmu_2,\ldots,\bmu_K$.
\item
  \code{LTSigma} is a matrix with dimension $K\times p(p+1)/2$ containing
  low triangular matrices of $\bSigma_1,\bSigma_2,\ldots,\bSigma_K$.
\item
  \code{lab} is a vector with length $n$ containing labels of observations.
  If \code{lab=NULL}, then unsupervised clustering is performed, otherwise
  semi-supervised clustering is performed where labels can be
  $1, 2,\ldots,K$ for cluster-known data and $0$ for cluster-unknown data.
\end{itemize}

